\chapter{Particle Flow Calorimetry}

\par This chapter will essentially cover all the details I learned from the efforts of CALICE collaboration carried on in recent 10 years. This chapter is not necessary to be consistent with the first chapter, since it is not prepared as my real dissertation.

\section{Calorimeter}
\par In this section, the basic physics including the electromagnetic and hadronic shower will be covered in detail. The role and limitations of calorimeters in high-energy physics will be discussed and introduce the general idea of particle flow calorimetry.
\subsection{Electromagnetic shower}

\subsection{Hadronic shower}

\subsection{Particle flow calorimetry}

\section{Particle flow detector and calorimeter technologies}
\par In this section, descriptions of the whole ILD and SiD calorimeters systems will be covered. Also, the baseline design of the sub-detector systems is introduced. 

\subsection{The ILD and SiD calorimeter systems}

\subsection{Electromagnetic calorimeter}

\subsection{Hadronic calorimeter}

\section{Calorimeter prototype performance in the Test-beam}
\par A overview of the test-beam will be given, and the corresponding results will be discussed in this section. The relationship between this section and the last one so far is not so clear to me and needs more thinkings.

\subsection{Overview of the test-beam}

\section{AHCAL: from physics prototype to engineering prototype}

\section{Calibration of AHCAL}

\section{Tests of GEANT4 shower simulation models}
\subsection{Physics lists}
\subsection{Electrons in SiW ECAL}
\subsection{Electrons in SciFe HCAL}
\subsection{Hadrons in SiW HCAL}
\subsection{Hadrons shower shapers in scintillator AHCAL}
\subsection{Charge track segments in hadronic shower}
\subsection{Shower evolution with time}
\subsection{Software compensation}

\section{Particle flow algorithm}















